\section{Tools and Frameworks}
\label{Chapters/Background/Tools-and-Frameworks}

\subsection{Open Services Gateway Initiative Framework}
\label{Chapters/Background/OSGi}
\texttt{OSGi}~\cite{OSGi} is dynamic module and service system platform for Java. Application components (distributed as \texttt{bundles}) can be remotely managed without requiring a reboot of the entire container. A package management system enables developers to package public and private APIs within the same bundle, but only expose public APIs at runtime. The dynamic service system allows the components to discover the addition of new services and act accordingly.

\textit{Eugene} mainly uses \texttt{OSGi} as a convenient deployment system, but also to enforce the principle of information hiding (see~\Cref{Chapters/Implementation/Overall-Principles}) through the use of private APIs that are hidden at runtime.

\subsection{Java Agent DEvelopment Framework}
\texttt{JADE}~\cite{JADE} is a software framework for developing distributed, multi-agent software systems in Java. \texttt{JADE} operates as a set of nodes which together form a platform (cluster); the nodes can be separated by a network layer, but standalone mode is also available. Each node hosts a set of \texttt{Agents} and is responsible for managing asynchronous communication between them and the other nodes. Each \texttt{Agent} (encapsulated in \texttt{jade.core.Agent} class) operates in a separate thread of control, therefore allowing independent, preemptive behaviour. \texttt{Agents} in \texttt{JADE} are not units of behaviour; they are only responsible for scheduling the different many behaviours that an \texttt{Agent} can have and exposing a set of primitives for sending and receiving messages (message queues). The actual behaviours are encapsulated in \texttt{jade.core.behaviours.Behaviour} class and its many subclasses, designed as building blocks for forming more sophisticated behaviours.

\texttt{JADE} exposes an \texttt{OSGi} service, therefore each \texttt{JADE} node can be run inside an \texttt{OSGi} container. However, we only use \texttt{OSGi} at deployment time, not during testing; that is because no part of \textit{Eugene} directly depends on \texttt{OSGi} and due to overall difficulty of testing code inside an \texttt{OSGi} container.

\subsection{Simple Logging Facade for Java}
\texttt{SLF4J}~\cite{SLF4J} is an abstraction for various logging frameworks that allows the user to plug in the desired implementation at deployment time. Combined with \texttt{LOGBack}~\cite{logback} implementation, it is a very powerful and versatile logging framework. Among the most useful features is the ability to add \textit{Markers} to a log entry to indicate the type of the event being logged and redirect events to different files based on the \textit{Markers}.
